\documentclass[12pt,a4paper]{article}
\usepackage[a4paper, left=3cm, right=3cm, top=3cm, bottom=3cm]{geometry}
\hyphenpenalty=10000
\renewcommand{\tablename}{Tabela}
\usepackage{graphicx} % Required for inserting images
\usepackage{array}
\usepackage{caption}
\usepackage{microtype}
\usepackage[utf8]{inputenc}
\usepackage[T1]{fontenc}
\usepackage[polish]{babel}
\usepackage{listings}
\usepackage{xcolor}
\sloppy

\definecolor{darkgreen}{rgb}{0,0.70,0}
\definecolor{lightblue}{rgb}{0.0,0.42,0.91}
\definecolor{grey}{rgb}{0.60, 0.60, 0.60}

\title{Analiza wariancji}
\author{Weronika Przysiężna}
\date{Marzec 2025}

\setlength{\parindent}{0pt}
\lstset{
  language=Python,
  aboveskip=1em,
  basicstyle=\ttfamily\footnotesize,
  breaklines=true,
  abovecaptionskip=-6pt,
  captionpos=b,
  escapeinside={\%*}{*)},
  frame=single,
  numberstyle=\tiny,
  keywordstyle=\color{lightblue},
  commentstyle=\color{grey},
  stringstyle=\color{darkgreen},
  showstringspaces=false,
  tabsize=4,
  literate={{ą}{{\k a}}1
           {Ą}{{\k A}}1
           {ż}{{\. z}}1
           {Ż}{{\. Z}}1
           {ź}{{\' z}}1
           {Ź}{{\' Z}}1
           {ć}{{\' c}}1
           {Ć}{{\' C}}1
           {ę}{{\k e}}1
           {Ę}{{\k E}}1
           {ó}{{\' o}}1
           {Ó}{{\' O}}1
           {ń}{{\' n}}1
           {Ń}{{\' N}}1
           {ś}{{\' s}}1
           {Ś}{{\' S}}1
           {ł}{{\l}}1
           {Ł}{{\L}}1}
}

\begin{document}

\maketitle
\newpage
\section{Teoretyczne wprowadzenie do analizy wariancji}

\subsection{Wstęp}
Analiza wariancji jest analizą statystyczną, która wykrywa różnice między dwiema lub więcej grupami określonymi dla pojedynczego czynnika lub zmiennej niezależnej. Identyfikuje ona zmienność lub wariancję pomiędzy obserwacjami przypisując ją różnym źródłom, które (po odpowiednim przetestowaniu) wskazują, czy zaobserwowane różnice między średnimi grupowymi są prawdopobnie rzeczywiste, czy jedynie wynikiem przypadku. \cite{witte2010statistics}\\
\\W naszym badaniu skupimy się na wykrywaniu różnic przy pomiarach na różnych osobach.

\subsection{Założenia dotyczące jednoczynnikowej analizy wariancji (ANOVA) z jednym czynnikiem międzyobiektowym \cite{orourke2005sas}}

\begin{itemize}
    \item \textbf{Zmienna zależna mierzona na skali ilościowej}: zmienna zależna powinna być zmienną ilościową (na poziom interwałowym lub ilorazowym).
    \item \textbf{Losowość i niezależność obserwacji}: nie ma związku między obserwacjami w każdej grupie lub między samymi grupami, a w każdej grupie są różni uczestnicy badania i żaden uczestnik nie należy do więcej niż jednej grupy;  uczestnicy badania są dobierani losowo.
    \item \textbf{Równoliczność obserwacji w grupach}: poszczególne kategorie zmiennej niezależnej powinny być statystycznie równoliczne (aby sprawdzić, czy analizowane grupy różnią się istotnie statycznie pod względem liczebności, można zastosować test zgodności Chi-kwadrat).
    \item \textbf{Rozkład normalny}: rozkład wyników w analizowanych grupach jest zbliżony do rozkładu normalnego (oceny tego założenia można dokonać stosując test Kołomogorowa-Smirnova lub Shapiro-Wilka).
    \item \textbf{Wariancje w grupach są jednorodne (homogeniczność wariancji)}: zmienność w każdej porównywanej grupie powinna być podobna; jeśli wariancje różnią się między grupami, to można zastosować test Welcha lub Browna-Forsythe'a, które wprowadzają poprawkę na nierówne wariancje do statystyki F.

\end{itemize}

\newpage
\subsection{Jednoczynnikowa ANOVA}

\subsubsection{Suma kwadratów SS (ang. Sum of Squares)}
Wariancja próby mierzy zmienność w dowolnym zbiorze obserwacji poprzez obliczenie sumy kwadratów odchyleń od ich średniej:
$$SS=\sum(X-\overline{X})^2.$$
Następnie suma kwadratów $SS$ jest dzielona przez liczbę stopni swobody $n-1$:
$$s^2=\frac{SS}{df},$$
gdzie:
\begin{itemize}
    \item $\overline{X}$ - średnia próby,
    \item $s^2$ - wariancja próby,
    \item $df=n-1$ - stopnie swobody.
\end{itemize}

\subsubsection{Średnia kwadratów MS (ang. Mean Square)}
Średnia kwadratów to oszacowanie wariancji uzyskane przez podzielenie sumy kwadratów $SS$ przez liczbę stopni swobody $n-1$.
Ogólny wzór na oszacowanie wariancji ma postać:
$$MS=\frac{SS}{df},$$
gdzie:
\begin{itemize}
    \item $MS$ - średnia kwadratów,
    \item $SS$ - suma kwadratów odchyleń od średniej,
    \item $df = n-1$ - liczba stopni swobody.
\end{itemize}

\newpage
\subsubsection{Wzory definicyjne na sumy kwadratów \cite{witte2010statistics}:}
\begin{enumerate}
    \item $SS_{total}$ - całkowita suma kwadratów odchyleń od średniej ogólnej (zmienność całkowita)
    $$SS_{total}=\sum(X-\overline{X}_{grand})^2.$$
    Równoważny wzór obliczeniowy:
    $$SS_{total}=\sum X^2-\frac{G^2}{N}.$$
    \item $SS_{between}$ - suma kwadratów odchyleń średnich grupowych od średniej ogólnej (zmienność między grupami)
    $$SS_{between}=\sum n(\overline{X}_{group}-\overline{X}_{grand})^2.$$
    Równoważny wzór obliczeniowy:
    $$SS_{between}=\sum \frac{T^2}{n}-\frac{G^2}{N}.$$
    \item $SS_{within}$ - suma kwadratów odchyleń indywidualnych wyników w grupie od średnich grupowych (zmienność wewnątrz grup)
    $$SS_{within}=\sum(X-\overline{X}_{group})^2.$$
    Równoważny wzór obliczeniowy:
    $$SS_{within}=\sum X^2-\sum\frac{T^2}{n}.$$
    \item Sprawdzamy dokładność obliczeniową weryfikując równość:
    $$SS_{total}=SS_{between}+SS_{within}$$
\end{enumerate}
Oznaczenia:
\begin{itemize}
    \item $X$ - pojedyncza wartość obserwowana,
    \item $\overline{X}_{group}$ - średnia dla danej grupy,
    \item $\overline{X}_{grand}$ - średnia ogólna dla całej próby,
    \item $T$ - suma wartości w grupie,
    \item $n$ - liczba obserwacji w grupie,
    \item $G$ - suma wartości dla wszystkich grup (suma ogólna),
    \item $N$ - całkowita liczba obserwacji (sumaryczna wielkość próby). 
\end{itemize}

\newpage
\subsubsection{Stopnie swobody ($df$):}
\begin{enumerate}
    \item $df_{total}=N-1$,
    \item $df_{between}=k-1$,
    \item $df_{within}=N-k$,
\end{enumerate}
gdzie:
\begin{itemize}
    \item $N$ - całkowita liczba obserwacji (sumaryczna wielkość próby),
    \item $k$ - liczba grup.
\end{itemize}
\vspace{2mm}
Sprawdzamy dokładność obliczeń weryfikując równość:
$$df_{total}=df_{between}+df_{within}.$$

\subsubsection{Wzory na średnie kwadratów \cite{witte2010statistics}:}
\begin{enumerate}
    \item $MS_{between}$ - średni kwadrat odchyleń między grupami (zmienność między średnimi dla grup)
    $$MS_{between}=\frac{SS_{between}}{df_{between}}$$
    \item $MS_{within}$ - średni kwadrat odchyleń wewnątrz grupy (zmienność wyników wewnątrz grupy; mierzy jedynie błąd losowy)
    $$MS_{within}=\frac{SS_{within}}{df_{within}}=MS_{error}$$
\end{enumerate}

\newpage
\subsubsection{Rozkład F-Snedecora}
Liczymy statystykę testową $F$ jako:
$$F=\frac{MS_{between}}{MS_{within}}$$
\\
Określamy obszar krytyczny jako:
$$Q=\{F:F\geq F_{\alpha}\},$$
gdzie $F_{\alpha}$ jest wartością krytyczną odczytaną z tablic rozkładu F-Snedecora dla $(df_{between},df_{within})$ stopni swobody, czyli  $F(df_{between},df_{within})$.
\\
\begin{enumerate}
    \item Jeżeli $F\in Q$ ($F\geq F_{\alpha}$), to odrzucamy hipotezę zerową $H_0$ na korzyść hipotezy alternatywnej $H_A$ i wnioskujemy, że badane średnie nie są sobie równe.
    \item Jeżeli $F\not\in Q$ ($F< F_{\alpha}$), to nie ma podstaw do odrzucenia hipotezy zerowej $H_0$ i wnioskujemy, że badane średnie są równe.
\end{enumerate}


\newpage
\subsubsection{Testy post-hoc (test HSD Tukey'a)}

\subsubsection{Obliczenie siły efektu (d Cohen'a)}

\newpage
\section{Praktyczne zastosowanie analizy wariancji}
W celu zilustrowania praktycznego zastosowania analizy wariancji przeprowadziłam badania na dwóch zestawach danych:

\begin{enumerate}
    \item Dane uzyskane za pośrednictwem ankiet, których celem jest analiza hipotetycznych zależności między czasem poświęcanym tygodniowo na aktywność fizyczną a nawykami oraz jakością życia.
    \item Dane udostępnione na platformie Kaggle.com, umożliwiające zbadanie potencjalnych zależności między [...] \footnote{Dane wykorzystane w analizie pochodzą z platformy Kaggle.com i zostały syntetycznie wygenerowane. Oznacza to, że nie są wynikiem rzeczywistych pomiarów, lecz zostały stworzone na podstawie określonych założeń i symulacji.}
\end{enumerate}
W dalszej części omówię proces pozyskania obu zbiorów danych oraz przedstawię w jaki sposób przeprowadziłam analizę statystyczną danych.

\vspace{8mm}

\subsection{Badanie zależności między czasem poświęcanym na aktywność fizyczną a jakością życia}

\vspace{5mm}

\subsubsection{Metodologia pozyskiwania danych ankietowych}
Dane zostały zebrane za pomocą ankiety utworzonej w Google Forms i udostępnionej w mediach społecznościowych, w tym na grupach na Facebooku zrzeszających osoby aktywne fizycznie oraz studentów, a także na prywatnych profilach na Facebooku i Instagramie, skierowanych do znajomych.

\vspace{3mm}

Grupą badaną są młodzi dorośli w wieku 18-35 lat, posługujący się językiem polskim, posiadający dostęp do internetu oraz do mediów społecznościowych.

\newpage
Pytaliśmy uczestników badania o ich wiek, płeć oraz odpowiedzi na poniższe pytania:
\begin{enumerate}
    \item {Ile czasu średnio tygodniowo poświęcasz na aktywność fizyczną? \\ (np. siłownia/rower/bieganie/taniec/joga)}
    \item Ile czasu średnio tygodniowo spędzasz na pozasportowych spotkaniach towarzyskich? Chodzi o spotkania ze znajomymi poza szkołą/miejscem pracy.
    \item {Ile czasu średnio dziennie spędzasz przed ekranem poza pracą/szkołą? \\ 
    (TV, komputer, telefon)}
    \item Ile czasu średnio śpisz w ciągu doby?
    \item {Ile razy chorowałeś/chorowałaś w ciągu ostatnich 12 miesięcy? \\
    (przeziębienie, grypa, choroby inne niż przewlekłe)}
    \item Na ile (w skali 1 - 10) oceniasz swoje ogóle poczucie szczęścia i zadowolenia z życia?
\end{enumerate}
\vspace{4mm}

Możliwe do wyboru odpowiedzi dla poszczególnych pytań prezentuje tabela:
\begin{table}[h]
    \centering
    \begin{tabular}{|l|l|l|}
        \hline \textbf{Pytania 1, 2, 3, 4} & \textbf{Pytanie 5} & \textbf{Pytanie 6} \\ \hline 
        mniej niż 1 godzinę & 0 - wcale &\\ \hline
        1 - 2 godziny & 1 raz & 1 - bardzo źle \\ \hline
        2 - 3 godziny & 2 razy & 2 \\ \hline
        3 - 4 godziny & 3 razy & 3 \\ \hline
        4 - 5 godzin & 4 razy & 4 \\ \hline
        5 - 6 godzin & 5 razy & 5 \\ \hline
        6 - 7 godzin & 6 razy & 6 \\ \hline
        7 - 8 godzin & 7 razy & 7 \\ \hline
        8 - 9 godzin & 8 razy & 8 \\ \hline
        więcej niż 9 godzin & 9 razy & 9 \\ \hline
         & 10 - 10 lub więcej niż 10 razy & 10 - bardzo dobrze \\ \hline
    \end{tabular}
    \caption*{Tabela 1}
\end{table}

\newpage
\subsubsection{Wyniki ankiety}
W wyniku ankiety uzyskaliśmy 597 odpowiedzi.

\vspace{2mm}
Uzyskane dane wymagały przetworzenia w formę umożliwiającą ich dalszą analizę, tak aby odpowiedzi na pytania 1-6 były zapisane jako dane numeryczne. W tym celu opracowałam program w języku Python, który automatycznie wykonuje tę operację:

\vspace{3mm}
\begin{lstlisting}
import pandas as pd

dane = pd.read_excel('dane.xlsx')       # wczytanie pliku

dane = dane.iloc[:, 1:]                 # usunięcie 1. kolumny

dane.columns = ['wiek', 'płeć', 
                'akt_fiz', 'spot_tow',
                'ekran', 'sen', 
                'choroby', 'szczescie'] # zmiana nazw kolumn

dane = dane[dane['wiek'] <= 35]         # usunięcie wierszy, 
                                        # w których wiek > 35

dane = dane.reset_index(drop=True)      # resetowanie indeksów

# funkcja do konwersji wartosci w kolumnach
def konwertuj(wartosc):
    if 'mniej niż' in wartosc:
        return 0
    elif 'więcej niż' in wartosc:
        return 9
    elif '-' in wartosc:
        return int(wartosc.split('-')[0].strip())
    elif 'Kobieta' in wartosc:
        return 'K'
    elif 'Mężczyzna' in wartosc:
        return 'M'
    elif 'Inne' in wartosc:
        return 'I'

dane['płeć'] = dane['płeć'].astype(str).apply(konwertuj)
dane['akt_fiz'] = dane['akt_fiz'].astype(str).apply(konwertuj)
dane['spot_tow'] = dane['spot_tow'].astype(str).apply(konwertuj)
dane['ekran'] = dane['ekran'].astype(str).apply(konwertuj)
dane['sen'] = dane['sen'].astype(str).apply(konwertuj)

print(dane)
\end{lstlisting}

\newpage W rezultacie pozostały 522 wiersze z odpowiedziami.

\begin{lstlisting}
      wiek płeć  akt_fiz  spot_tow  ekran  sen  choroby  szczescie
0      31    M        9         7      4    6        2          7
1      25    M        2         3      2    6        5          7
2      29    M        2         5      5    7        4          3
3      25    M        6         0      9    6        3          4
4      29    M        3         2      7    7        1         10
..    ...  ...      ...       ...    ...  ...      ...        ...
517    25    M        6         7      3    8        1          9
518    32    M        0         3      3    5        1          7
519    32    M        1         9      1    7        5          8
520    25    M        0         2      3    8        1          5
521    25    M        0         0      9    6        2          4
\end{lstlisting}

\subsubsection{Wstępna analiza zebranych danych}

Wywołanie \texttt{describe} dla wybranych kolumn pozwoliło na uzyskanie zbiorczego opisu statystycznego tych zmiennych. 
\begin{lstlisting}
    print(dane[['wiek', 'płeć', 'akt_fiz', 'spot_tow', 'ekran', 
    'sen', 'choroby', 'szczescie']].describe(include='all'))
\end{lstlisting}
W wyniku wywołania powyższej linijki kodu otrzymujemy:
\begin{lstlisting}[basicstyle=\ttfamily\fontsize{7}{8}\selectfont\color{black},
  keywordstyle=\color{black},
  commentstyle=\color{black},
  stringstyle=\color{black},
  frame=single]
              wiek płeć     akt_fiz    spot_tow       ekran         sen     choroby   szczescie
count   522.000000  522  522.000000  522.000000  522.000000  522.000000  522.000000  522.000000
unique         NaN    3         NaN         NaN         NaN         NaN         NaN         NaN
top            NaN    M         NaN         NaN         NaN         NaN         NaN         NaN
freq           NaN  373         NaN         NaN         NaN         NaN         NaN         NaN
mean     26.168582  NaN    3.340996    3.105364    3.894636    6.358238    2.055556    6.281609
std       3.857631  NaN    2.825475    2.753189    2.415245    1.185331    1.784659    2.155920
min      18.000000  NaN    0.000000    0.000000    0.000000    0.000000    0.000000    1.000000
25%      24.000000  NaN    1.000000    1.000000    2.000000    6.000000    1.000000    5.000000
50%      26.000000  NaN    3.000000    2.000000    3.000000    6.000000    2.000000    7.000000
75%      29.000000  NaN    5.000000    5.000000    5.000000    7.000000    3.000000    8.000000
max      35.000000  NaN    9.000000    9.000000    9.000000    9.000000   10.000000   10.000000
\end{lstlisting}

Funkcja zwróciła m.in.:
\begin{itemize}
    \item liczność (\texttt{count}) – liczbę dostępnych wartości,
    \item średnią (\texttt{mean}) – wartość przeciętną,
    \item odchylenie standardowe (\texttt{std}) – miarę rozproszenia danych,
    \item minimalną i maksymalną wartość (\texttt{min}, \texttt{max}),
    \item percentyle (\texttt{25\%}, \texttt{50\%} (mediana), \texttt{75\%}), które pokazują, jak dane są rozłożone.
    \item liczność (\texttt{count}),
    \item liczba unikalnych wartości (\texttt{unique}),
    \item najczęściej występująca wartość (\texttt{top}),
    \item częstość występowania tej wartości (\texttt{freq}).
\end{itemize}

Dzięki tym informacjom możliwe było uzyskanie pierwszego wglądu w rozkład danych.

\newpage
\subsubsection{Wstępna analiza zebranych danych}





\newpage
\subsection{Badanie zależności między [...]}
\subsubsection{Opis danych}
\subsubsection{Analiza statystyczna zebranych danych}


\newpage


\section{Bibliografia}

Pozycje \cite{gitkkozlowski}, \cite{pandas}, \cite{statsmodels} służyły jako pomoc przy pisaniu kodów w LaTeX i Python.
\bibliographystyle{plain}
\bibliography{bibliografia}



\end{document}
